\section{Revisão Bibliográfica}

O problema da mochila 0-1 pode ser formalizado como um problema de programação inteira binária. Dado um conjunto de $n$ itens numerados de $1$ a $n$, cada um com peso $w_i$ e valor $v_i$, e dado um limite de capacidade $W$, busca-se uma atribuição binária $x_i \in \{0,1\}$ (incluir ou não o item $i$) que maximize o valor total:

\begin{align*}
\text{Objetivo:} &\quad \max \sum_{i=1}^n v_i x_i \\
\text{Restrição:} &\quad \sum_{i=1}^n w_i x_i \le W \\
\text{Domínio:} &\quad x_i \in \{0,1\},\quad i=1,\dots,n
\end{align*}

Essa formulação reflete exatamente o desafio de maximizar o valor dos itens carregados sem exceder a capacidade da mochila. Em termos práticos, escolhe-se um subconjunto de itens cujo peso total não ultrapasse $W$ e cujo valor total seja o maior possível.

Quanto à complexidade computacional, o problema é intrinsecamente difícil. A versão de decisão (``existe subconjunto com valor pelo menos $V$ sem exceder $W$?'') é \textit{NP-completa}, e a versão de otimização é \textit{NP-difícil}. Mais especificamente, o problema da mochila 0-1 é \textit{fraco NP-difícil} (weakly NP-hard), pois admite algoritmos pseudo-polinomiais, como o de programação dinâmica com tempo $O(nW)$ \cite{pisinger2005}.

Esse tempo de execução é considerado pseudo-polinomial porque depende linearmente de $n$ e de $W$, mas exponencialmente do tamanho da entrada binária (já que $W$ pode ter $\log W$ bits). Por exemplo, dobrar o valor de $W$ dobra o tempo do algoritmo, mas o tamanho da entrada cresce apenas linearmente em $\log W$. Assim, $O(nW) = O(n2^{\log W})$ é exponencial no tamanho da entrada, o que mantém a consistência com a complexidade \textit{NP-difícil}.

Apesar disso, diversas abordagens exatas foram desenvolvidas. A mais clássica é a própria programação dinâmica. Outras envolvem técnicas como \textit{branch-and-bound}, onde limites obtidos por relaxação linear (como no problema da mochila fracionária) são usados para podar ramos de busca \cite{martello1990}. Heurísticas que partem de soluções fracionárias e as ajustam para soluções inteiras também têm sido efetivas \cite{pisinger2005}.

Outro algoritmo exato relevante é o \textit{meet-in-the-middle}, proposto por Horowitz e Sahni \cite{horowitz1974}, que, embora tenha complexidade exponencial em $n$, pode ser mais eficiente que a programação dinâmica quando $n$ é grande e $W$ é moderado.

Como alternativa à ausência de algoritmos exatos de tempo polinomial, foram propostos esquemas de aproximação. Em particular, o problema admite esquemas \textit{FPTAS} (Fully Polynomial-Time Approximation Schemes), que garantem aproximações com fator $1 + \varepsilon$ em tempo polinomial em $n$ e $1/\varepsilon$ \cite{ibarra1975,jin2019}.

Esses algoritmos baseiam-se na discretização dos valores ou pesos, e são capazes de produzir soluções arbitrariamente próximas do ótimo em tempo viável. Além disso, heurísticas metaheurísticas como algoritmos genéticos, colônia de formigas e enxame de partículas também são aplicadas em versões de larga escala, embora sem garantias formais de aproximação.

O problema possui ainda diversas variantes que ampliam sua complexidade e aplicabilidade:
\begin{itemize}
    \item \textbf{Mochila limitada} (bounded knapsack): permite múltiplas cópias limitadas de cada item ($x_i \in \{0, 1, \dots, c_i\}$);
    \item \textbf{Mochila ilimitada} (unbounded knapsack): número ilimitado de cópias de cada item ($x_i \in \mathbb{N}$);
    \item \textbf{Mochila multidimensional}: considera múltiplas restrições (como peso e volume);
    \item \textbf{Mochila múltipla ou de escolha múltipla}: os itens são agrupados em classes, sendo permitido escolher um item por classe;
    \item \textbf{Mochila quadrática e outras extensões}: envolvem funções objetivo não lineares;
    \item \textbf{Subset Sum}: caso especial onde $v_i = w_i$ para todos os itens, sendo exatamente \textit{NP-completo}, como listado por Karp \cite{karp1972}.
\end{itemize}

Essas variantes mostram a flexibilidade do problema da mochila como modelo para cenários reais e seu papel central na teoria da complexidade e otimização combinatória.
