\section{Conclusão}

O problema da mochila 0-1 é um dos pilares da otimização combinatória e da teoria da complexidade computacional,
servindo como referência para o estudo de algoritmos exatos, heurísticos e de aproximação.
Ao longo deste trabalho, revisamos sua formulação matemática, demonstramos sua classificação como problema NP-completo e analisamos diferentes abordagens de resolução,
desde métodos exatos como força bruta e programação dinâmica até heurísticas e algoritmos aproximados.

Observou-se que, embora algoritmos exatos sejam viáveis para instâncias pequenas ou moderadas,
sua aplicação em cenários reais de grande escala é limitada pelo crescimento exponencial do tempo de execução. 
Nesses casos, heurísticas e esquemas de aproximação tornam-se alternativas práticas, fornecendo soluções suficientemente boas em tempo reduzido.

Além de sua relevância teórica, o problema da mochila 0-1 possui ampla aplicação prática em áreas como logística, finanças,
corte de materiais e ciência da computação, sendo constantemente estudado e aprimorado.
O contínuo desenvolvimento de novas técnicas e algoritmos reforça sua importância tanto no meio acadêmico quanto no mercado.

Portanto, compreender o problema da mochila 0-1 e suas soluções contribui para a resolução eficiente de desafios complexos em diversas áreas do conhecimento.