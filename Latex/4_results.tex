\section{Resultados}
Nesta seção, apresentamos os resultados obtidos com as diferentes abordagens para resolver o problema da mochila 0-1.


\subsection*{Comparativo das Abordagens}

\begin{center}
\renewcommand{\arraystretch}{1.4}
\begin{tabular}{>{\bfseries}p{4cm} p{3.5cm} p{3.5cm} p{3.5cm}}
\toprule
Critério & Brute Force (Força Bruta) & Top-Down (Memoização) & Bottom-Up (Programação Dinâmica) \\
\midrule
Estratégia & Recursiva pura & Recursiva + cache & Iterativa com tabela \\
Estrutura usada & Árvore binária de chamadas & Recursão + memoização & Tabela bidimensional \\
Recorrência & \(T(n) = 2T(n-1) + 1\) & \(T(n, W) = T(n-1, W) + T(n-1, W - wt[n-1]) + 1\) & \(T(n, W) = n \cdot W \cdot c\) \\
Tempo & \(O(2^n)\) & \(O(n \cdot W)\) & \(O(n \cdot W)\) \\
Espaço & \(O(n)\) & \(O(n \cdot W) + O(n)\) & \(O(n \cdot W)\), ou \(O(W)\) com otimização \\
Subproblemas resolvidos & Todos (muitos repetidos) & Cada subproblema apenas uma vez & Todos iterativamente \\
Facilidade de implementação & Simples, porém ineficiente & Média, exige controle de cache & Mais verbosa, mas robusta \\
Eficiência prática & Muito baixa & Boa & Excelente \\
Melhor para & Problemas pequenos ou estudo conceitual & Casos moderados com repetição de subproblemas & Casos reais com grandes entradas \\
\bottomrule
\end{tabular}
\end{center}

\subsection*{Resumo dos Resultados}

\begin{itemize}
    \item A abordagem de \textbf{força bruta} serve bem para fins didáticos, mas é inviável para grandes entradas devido ao crescimento exponencial de chamadas.
    \item A \textbf{programação dinâmica top-down com memoização} melhora drasticamente a eficiência ao evitar recomputações, mantendo uma estrutura recursiva.
    \item A \textbf{abordagem bottom-up} é a mais eficiente em aplicações reais, pois evita recursão e permite otimização de espaço.
\end{itemize}

\subsection*{Recomendação de Uso}

\begin{itemize}
    \item Use \textbf{força bruta} apenas para instâncias muito pequenas ou fins educacionais.
    \item Use \textbf{top-down com memoização} para problemas moderados ou quando a estrutura recursiva for preferível.
    \item Use \textbf{bottom-up} para problemas grandes, onde desempenho e controle de memória são essenciais.
    \item Use \textbf{heurístico / guloso} Quando a valocidade for prioridade e correr o risco de perda de precisão for aceitável
\end{itemize}