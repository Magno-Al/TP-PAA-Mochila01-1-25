\section{Análise de complexidade e comparação}

A tabela a seguir resume a complexidade de cada abordagem:

\begin{table}[h]
  \centering
  \caption{Complexidade das Abordagens}
  \begin{tabular}{|c|c|c|}
    \hline
    \textbf{Abordagem} & \textbf{Complexidade de Tempo} & \textbf{Complexidade de Espaço} \\ \hline
    Força Bruta                        & \(O(2^n)\)                       & \(O(n)\)                        \\ \hline
    Programação Dinâmica - Bottom\_Up  & \(O(n \cdot W)\)                 & \(O(n \cdot W)\)                \\ \hline
    Programação Dinâmica - Top\_Down   & \(O(n \cdot W)\)                 & \(O(n \cdot W)\)                \\ \hline
  \end{tabular}
\end{table}



% Força Bruta ---------------------------------
\subsection{Brute Force (Força Bruta)}
\subsection*{Descrição}
A abordagem de força bruta resolve o problema recursivamente. Para cada item, ela considera duas opções: incluí-lo na mochila (
se houver capacidade) ou excluí-lo. O resultado é o máximo valor obtido entre essas duas opções.

\subsection*{Recorrência Algorítmica (Tempo)}
\[
T(n, W) = 
\begin{cases}
1 & \text{se } n = 0 \text{ ou } W = 0 \\
T(n - 1, W) + 1 & \text{se } wt[n - 1] > W \\
T(n - 1, W) + T(n - 1, W - wt[n - 1]) + 1 & \text{se } wt[n - 1] \leq W
\end{cases}
\]

\subsection*{Complexidade}
\textbf{Tempo:} \( O(2^N) \) \\
\textbf{Espaço:} \( O(N) \) (devido à pilha de chamadas recursivas)




% Top Down ---------------------------------
\subsection{Top-Down Dynamic Programming (com Memoização)}
\subsection*{Descrição}
Esta abordagem otimiza a força bruta utilizando memoização para evitar recálculos. Antes de computar um subproblema, 
verifica-se se o resultado já está armazenado. Se sim, retorna o valor armazenado; caso contrário, calcula e armazena.

\subsection*{Recorrência Algorítmica (Tempo)}
Embora a estrutura seja igual à da força bruta, a memoização garante que cada subproblema \( (n, W) \) seja computado no máximo uma vez.
Assim, a complexidade é:

\[T(n, W) = \begin{cases}
1 & \text{se } n = 0 \text{ ou } W = 0 \\
1 + T(n - 1, W) & \text{se } wt[n - 1] > W \\
1 + \text{(valor já memoizado ou computado uma vez)} & \text{caso geral}
\end{cases}\]

\subsection*{Complexidade}
\textbf{Tempo:} \( O(N \cdot W) \) \\
\textbf{Espaço:} \( O(N \cdot W) \) para a tabela de memoização + \( O(N) \) para a pilha de chamadas




% Bottom Up ---------------------------------
\subsection{Bottom-Up Dynamic Programming}
\subsection*{Descrição}
A abordagem bottom-up preenche uma tabela iterativamente, resolvendo os subproblemas menores primeiro. A solução é construída de baixo para cima, começando com 0 itens e 0 capacidade.

\subsection*{Recorrência Algorítmica (Tempo)}
O preenchimento da tabela envolve dois laços aninhados:

\[T(n, W) = O(n \cdot W)\]

\subsection*{Complexidade}
\textbf{Tempo:} \( O(N \cdot W) \) \\
\textbf{Espaço:} \( O(N \cdot W) \). Pode ser otimizado para \( O(W) \) se apenas a linha anterior for necessária.




% Heuristica (Guloso) ---------------------------------
\section{Algoritmos de Aproximação e Heurísticas para o Problema da Mochila 0/1}

O problema da mochila 0/1 pertence à classe dos problemas NP-completos. Essa classificação implica que, até o presente momento,
não se conhece um algoritmo que seja capaz de resolvê-lo de forma exata, em tempo polinomial, para todas as possíveis instâncias.
Consequentemente, à medida que o número de itens cresce, os algoritmos exatos, como a solução recursiva simples ou mesmo a programação dinâmica,
se tornam computacionalmente inviáveis, especialmente em sistemas com recursos limitados ou em aplicações que exigem respostas rápidas.

Diante dessas limitações, surgem como alternativa os algoritmos de aproximação e heurísticas, que têm por objetivo encontrar soluções suficientemente boas dentro de um tempo de execução consideravelmente reduzido. Tais algoritmos não garantem a obtenção da solução ótima, mas frequentemente produzem resultados aceitáveis com significativa economia de tempo computacional.

Uma das heurísticas mais conhecidas e frequentemente utilizadas para o problema da mochila é a abordagem gulosa (greedy), baseada na razão valor/peso de cada item.
A estratégia dessa heurística é selecionar prioritariamente os itens que oferecem o maior “retorno por unidade de peso”, o que, intuitivamente,
tende a maximizar o valor total carregado na mochila.


\begin{algorithm}[H]
  \caption{MochilaGulosa(capacidade, pesos, valores, n)}
  \begin{algorithmic}[1]
  \State Para cada item $i$, calcule o benefício: $\text{benefício}_i = \frac{\text{valores}[i]}{\text{pesos}[i]}$
  \State Ordene os itens em ordem decrescente de benefício
  \State Inicialize a mochila como vazia e $valor\_total \gets 0$
  \For{cada item $i$ na lista ordenada}
      \If{$pesos[i] \leq capacidade\_restante$}
          \State Adicione o item $i$ à mochila
          \State $valor\_total \gets valor\_total + valores[i]$
          \State $capacidade\_restante \gets capacidade\_restante - pesos[i]$
      \EndIf
  \EndFor
  \State \Return $valor\_total$
  \end{algorithmic}
\end{algorithm}

O funcionamento do algoritmo guloso pode ser descrito nas seguintes etapas:

\begin{itemize}
  \item Cálculo da razão valor/peso: Para cada item disponível, é calculada a razão entre seu valor e seu peso, representando a eficiência relativa do item.
  \item Ordenação dos itens: Os itens são ordenados em ordem decrescente de razão valor/peso.
  \item Seleção dos itens: Os itens são inseridos na mochila, um a um, na ordem estabelecida, desde que o peso total acumulado não ultrapasse a capacidade máxima permitida (W).
\end{itemize}

Embora essa estratégia seja eficiente do ponto de vista computacional, apresentando complexidade temporal de O(n log n) devido à etapa de ordenação,
ela possui limitações estruturais. 

Em particular, o algoritmo guloso não é garantidamente ótimo para a versão 0/1 da mochila, 
pois nesta não é permitido incluir frações de um item. Assim, decisões locais baseadas na razão valor/peso podem impedir que o algoritmo alcance
a combinação globalmente ótima de itens. 

Por exemplo, pode haver casos em que deixar de fora um item muito eficiente localmente permite 
incluir outros dois menos eficientes, mas que juntos fornecem um valor total maior. Essa heurística, no entanto, se mostra bastante eficaz quando 
aplicada à variante racionária do problema da mochila, na qual é permitido incluir partes dos itens. Nesse caso, o algoritmo guloso fornece 
uma solução ótima e é amplamente adotado como método principal.

Heurísticas gulosas são ferramentas importantes em contextos onde o tempo de execução 
é um fator crítico e uma solução próxima da ideal é suficiente, mas é essencial compreender suas limitações para aplicá-las de forma adequada, considerando a
natureza do problema e os requisitos de qualidade da solução.